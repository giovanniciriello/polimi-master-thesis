\chapter{Conclusions and future developments}
\label{ch:conclusions}

The methodology described in this paper is the result of many iterations and experiments, which only in the final phase led to concrete results supported by figures. However, now that the methodology has settled in, there is nothing to stop it from being reproduced in the shortest possible time to detect anomalies in real time.

The interesting thing about the approach I used is that it can be used with future data without the need to re-train all the models. Since we are dealing with S-AIS data - where the S stands for Satellite, one can imagine performing this type of process during the phase of direct reception of messages directly from satellites, once a route is completed by a monitored vessel.

Moreover, the detected anomalies described in the \ref{sec:testing-importance} section are the result obtained from a very large data set, but one that is tiny considering the amount of AIS messages available for analysis and imagining a computational power far greater than that used in this work. Indeed, with larger data sets, one could obtain many more results, more accurate models, and as a direct result, more meaningful interpretation of the numbers.