\chapter*{Introduction}
\label{ch:introduction}

The work presented in this paper is related to the ARCOS project and aims to identify potential anomalies in the routes of maritime transport in the Arctic region.

The chapter \ref{ch:stateOfTheArt} gives an overview of the background regarding the ARCOS project and its goals, describes the AIS maritime messaging standard with a focus on satellite messaging (S- AIS), and also a description of the anomaly detection techniques currently in use with a focus on DBSCAN, which will be used to cluster trips.

The chapter \ref{ch:methodology} contains all the methodology that has been used, starting with the raw AIS data, moving through data cleaning techniques, and ending with the generation of trip entities using a logical clustering algorithm. It continues with the selection and computation of trip features for analysis (features engineering) and the implementation of the DBSCAN clustering algorithm. Finally, the methodology includes data analysis techniques useful for understanding why some trips are considered outliers and which ones are significant or not.

The chapter \ref{ch:testing} describes the results obtained on a dataset containing more than 95 million messages trying to allow an interpretation of the most relevant anomalies.