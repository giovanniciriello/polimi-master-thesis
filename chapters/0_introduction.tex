\chapter*{Introduction}
\label{ch:introduction}

Il lavoro presentato in questo documento è collegato al progetto ARCOS e ha come obiettivo la rilevazione di potenziali anomalie nei tragitti effettuati da mezzi di trasporto marittimi nella zona dell'Artico.

Nel capitolo \ref{ch:stateOfTheArt} si fa una panoramica del contesto riguardante il progetto ARCOS e i suoi obiettivi, si descrive lo standard di messaggi marittimi AIS con un focus sui messaggi satellitari (S-AIS), e anche una descrizione delle tecniche di anomaly detection attualmente utilizzate, con un focus sul DBSCAN, che verrà utilizzato per clusterizzare i viaggi.

Il capitolo \ref{ch:methodology} include la completa metologia utilizzata, partendo dai dati AIS grezzi, iniziando con tecniche di data cleaning e procedendo con la generazione delle entitià tragitti tramite un algoritmo di raggruppamento logico. Procedendo ulteriormente con la selezione e la computazione delle feature dei viaggi da analizzare (features engineering) e l'implementazione dell'algoritmo di clustering DBSCAN. Infine la metodologia comprende tecniche di data analysis utili a comprendere i motivi per cui alcuni viaggi sono considerati outlier e quali sono significativi o meno.

Nel capitolo \ref{ch:testing} si descrivono i risultati ottenuti con un dataset contenente più di 95 milioni di messaggi cercando di dare un'interpretazione alle anomalie più rilevanti.
